\section{Methodological approach}\label{sec:approach}

\subsection{Data Sources}

In our research we will use the following data sources:

\begin{itemize}
	\item Academic ranking of world universities(ARWU) %write citation
	\item OECD Education Statistics(OECD iLibrary) %write citation
\end{itemize}

\subsection{Main Variables}
We will use a total of five variables for constructing the ranking analysis. Those variables will be:
\begin{enumerate}
	\item Overall score (500 universities)
	\item Top clinical medicine and pharmacy
	\item Social sciences
	\item Natural sciences and mathematics
	\item Engineering/Technology and computer science
\end{enumerate}

\subsection{Indicators and Weights}
\begin{table}[h]
\begin{tabular}{lllll}
Criteria               & Indicator                                                                                                                                                            & Weight                                              & \multicolumn{2}{l}{\multirow{5}{*}{}} \\
Quality of Education   & Alumniof an institution winning Nobel Prizes and Fields Medals                                                                                                       & 10\%                                                & \multicolumn{2}{l}{}                  \\
Quality of Faculty     & \begin{tabular}[c]{@{}l@{}}Staff of an institution winning Nobel Prizes and Fields Medals\\ Highly cited researchers in 21 broad subject categories\end{tabular}     & \begin{tabular}[c]{@{}l@{}}20\%\\ 20\%\end{tabular} & \multicolumn{2}{l}{}                  \\
Research Output        & \begin{tabular}[c]{@{}l@{}}Papers published in Nature and Science\\ Papers indexed in Science Citation Index-expanded and\\ Social Science Citation Index\end{tabular} & \begin{tabular}[c]{@{}l@{}}20\%\\ 20\%\end{tabular} & \multicolumn{2}{l}{}                  \\
Per Capita Performance & Per capita academic performance of an institution                                                                                                                    &   10\%                                                  & \multicolumn{2}{l}{}                 
\end{tabular}
\end{table}

\subsection{Level of analysis}
The level of analysis we will use for this project given the type of data is a macro level analysis.
\subsection{Statistical techniques to be used}
We will use correlation and classification.